%///////////////////////////////////////////////////////////////////////////////
%//  _   _        _                            _                                
%// | | | |      |_|                          | |                               
%// | |_| |_____  _ _____           _     _ __| |                               
%// | |_  | ___ || |  _  \  _____  \ \  / // _  |                               
%// | | | | ____|| | |_| | (_____)  \ \/ /( (_| |                               
%// |_| |_|_____)|_|___  |           \__/  \____|                               
%//                  __| | Haute Ecole d'Ingenieurs                             
%//                 |___/  et de Gestion - Vaud                                 
%//                                                                             
%// @title    M�moire de dipl�me: Softwares                                     
%// @context  Travail de dipl�me                                                
%// @author   Y. Chevallier <nowox@kalios.ch>                                   
%// @file     software.tex                                                      
%// @language ASCII/C                                                           
%// @svn      $Id$                
%///////////////////////////////////////////////////////////////////////////////
\chapter*{Logiciels}
Bien qu'il ne soit pas courant de faire r�f�rence aux logiciels utilis�s durant le projet. Il est d'avis de l'auteur que l'air de l'informatique am�ne avec elle une panoplie consid�rable d'outils informatiques permettant de l'�laboration de l'aspect typographique de ce document � la confection des diff�rentes figures. 

\begin{description}
	\item{[1]} Vim Improved 6.3 --- Digne successeur de \emph{vi} l'�diteur multiplateforme apparu avec les syst�mes \emph{unix}, il supporte l'�dition d'un grand nombre de type de fichiers\\
		\url{http://www.vim.org/}
	\item{[2]} Subversion --- Subversion est un logiciel open source de gestion de version. L'ensemble des documents relatifs � ce dipl�me sont sous contr�le de version. \\
		\url{http://subversion.tigris.org/} 
	\item{[3]} Adobe Illustrator CS2 --- Une r�f�rence en mati�re de logiciel d'�dition d'images vectorielles, Illustrator � son d�but r�serv� aux utilisateurs de Macintosh est � pr�sent couramment utilis� par les professionnels de l'�dition sur des machines tournant sur MS Windows. L'ensemble des figures pr�sent�es dans ce documents ont �t� soit cr�e, soit retravaill�e avec ce programme. \\
		\url{http://www.adobe.fr/products/illustrator/main.html}
	\item{[4]} Adobe Photoshop CS2 --- La r�putation de Photoshop n'est plus � faire en mati�re de retouche d'images. Il a servi, entre autres, � retoucher les images de l'interface web de LonTouch. \\
		\url{http://www.adobe.fr/products/photoshop/main.html}
	\item{[5]} Miktex --- Il s'agit d'une distribution de \LaTeX pour windows. \LaTeX permet la redactions de documents tel que celui que vous avez entre les mains en ce moment. Il est principalement utilis� dans des contextes d'articles scientifiques et de livres techniques. Sa prise en main difficile ne le rend pas accessible � tout public.
	\item{[6]} DbDesigner 4 --- Ce logiciel libre permet l'�dition d'une base de donn�es avec une grande facilit�. \\
		\url{http://fabforce.net/dbdesigner4/} 
	\item{[7]} NL220 --- Il s'agit d'un outil d'administration pour les r�seaux LonWorks. Son utilisation n�cessite un dongle hardware. \\
		\url{http://www.newron-system.com/fr/nl220.php}
	\item{[8]} Mozilla Firefox --- Le c�l�bre navigateur web de mozilla. \\
		\url{http://www.mozilla.org}
\end{description} 
