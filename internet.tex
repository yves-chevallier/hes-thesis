%///////////////////////////////////////////////////////////////////////////////
%//  _   _        _                            _                                
%// | | | |      |_|                          | |                               
%// | |_| |_____  _ _____           _     _ __| |                               
%// | |_  | ___ || |  _  \  _____  \ \  / // _  |                               
%// | | | | ____|| | |_| | (_____)  \ \/ /( (_| |                               
%// |_| |_|_____)|_|___  |           \__/  \____|                               
%//                  __| | Haute Ecole d'Ingenieurs                             
%//                 |___/  et de Gestion - Vaud                                 
%//                                                                             
%// @title    M�moire de dipl�me: Liens Internet                                
%// @context  Travail de dipl�me                                                
%// @author   Y. Chevallier <nowox@kalios.ch>                                   
%// @file     cdrom.tex                                                         
%// @language ASCII/C                                                           
%// @svn      $Id$               
%///////////////////////////////////////////////////////////////////////////////
\chapter*{Liens Internet}
Un nombre important de renseignements n�cessaires � ce travail de dipl�me ont �t� trouv�s sur Internet. Il est alors utile de mentionner principaux sites Web visit�s. 
\begin{description}
	\item{[1]} 	\url{http://www.br-automation.com/} \\
		La soci�t� B\&R est celle qui fourni le PC � �cran tactile utilis� lors de ce travail.
	
	\item{[2]}  \url{http://www.debian.org/} \\
		Il s'agit de la distribution de Linux choisie pour LonTouch. 
		
	\item{[3]}  \url{http://developer.apple.com/internet/webcontent/xmlhttpreq.html} \\
		Le fonctionnement du composant XMLHttpRequest est expliqu� en d�tail � cette adresse.
		
	\item{[4]} 	\url{http://www.faqs.org/rfcs/} \\
		Les sp�cifications des diff�rents protocoles utilis�s (TCP, UDP, HTTP, \ldots) sont d�crit � cette adresse. 
		
	\item{[5]}  \url{http://fr.selfhtml.org/} \\
		Ce site Internet fourni bon nombre de renseignement pour r�aliser des documents HTML et DHTML
		
	\item{[6]} 	\url{http://fr.wikipedia.org/wiki/Accueil} \\
		Une r�f�rence en mati�re de source universelle d'informations
		
	\item{[7]} 	\url{http://www.google.ch/} \\
		L'indispensable moteur de recherche Internet. 
		
	\item{[8]} 	\url{http://www.lighttpd.net/} \\
		Est l'adresse du serveur HTTP choisis pour h�berger le WEBGUI. 
		
	\item{[9]} 	\url{http://www.lonmark.com/}	\\
		L'organisation s'occupant d'�tablir les standards LonWorks. 
		
	\item{[10]} \url{http://www.loytec.com/} \\ 
		Site de la soci�t� � l'origine des librairies Orion et Ossi 
		
	\item{[11]} \url{http://www.gnu.org/prep/standards/} \\
		Sont d�crit sur ce site les \emph{GNU coding standard}. 
		
	\item{[12]} \url{http://www.mozilla.org/} \\
		Lien vers la fondation � l'orgine du moteur de rendu Gecko et du navigateur Firefox.
		
	\item{[13]} \url{http://www.php.net/} \\
		Site officiel du langage de programmation PHP. 
		
	\item{[14]} \url{http://phrogz.net/ObjJob/} \\
		Une excellent source d'information concernant le DOM et le langage JavaScript
		
	\item{[16]} \url{http://www.newron-system.com/} \\
		Site Internet de la soci�t� � l'origine de l'outil de configuration LonWorks NL-220. 
		
	\item{[17]} \url{http://www.stack.nl/~dimitri/doxygen/} \\
		Est le site du projet Doxygen permettant l'auto-documentation de code source. 
\end{description}   
