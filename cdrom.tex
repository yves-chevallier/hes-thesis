%///////////////////////////////////////////////////////////////////////////////
%//  _   _        _                            _                                
%// | | | |      |_|                          | |                               
%// | |_| |_____  _ _____           _     _ __| |                               
%// | |_  | ___ || |  _  \  _____  \ \  / // _  |                               
%// | | | | ____|| | |_| | (_____)  \ \/ /( (_| |                               
%// |_| |_|_____)|_|___  |           \__/  \____|                               
%//                  __| | Haute Ecole d'Ingenieurs                             
%//                 |___/  et de Gestion - Vaud                                 
%//                                                                             
%// @title    M�moire de dipl�me: Contenu du CDROM                              
%// @context  Travail de dipl�me                                                
%// @author   Y. Chevallier <nowox@kalios.ch>                                   
%// @file     cdrom.tex                                                         
%// @language ASCII/C                                                           
%// @svn      $Id$                   
%///////////////////////////////////////////////////////////////////////////////
\chapter*{CDrom d'accompagnement}
Ce travail comportant une certaine quantit� de fichiers informatique, il � �t� jug� indispensable de joindre � ce document un cdrom compatible ISO9660. Voici la liste principale de son contenu: 

\begin{itemize}
	\item \texttt{doc/} --- Documents et notes relatifs au travail de dipl�me	 		
	\item \texttt{daemon/} --- Code source du d�mon \textsf{SERVEUR} et \textsf{CLIENT}
	\item \texttt{email/} --- Messages re�us et envoy�s � propos du dipl�me	 		
	\item \texttt{etc/} --- Param�tres de configuration Linux		
	\item \texttt{gi/} --- Les interfaces graphiques r�alis�es		
	\item \texttt{logo/} --- Les fichiers sources du logo LonTouch	  		
	\item \texttt{memoire/} --- Memoire au format \LaTeX	 						 
	\item \texttt{software/} --- Certains logiciels utilis�s durant ce travail  		
	\item \texttt{webgui/} --- Console d'administration Web pour LonTouch	 	 		
\end{itemize}

