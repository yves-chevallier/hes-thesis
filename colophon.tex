%///////////////////////////////////////////////////////////////////////////////
%//  _   _        _                            _                                
%// | | | |      |_|                          | |                               
%// | |_| |_____  _ _____           _     _ __| |                               
%// | |_  | ___ || |  _  \  _____  \ \  / // _  |                               
%// | | | | ____|| | |_| | (_____)  \ \/ /( (_| |                               
%// |_| |_|_____)|_|___  |           \__/  \____|                               
%//                  __| | Haute Ecole d'Ingenieurs                             
%//                 |___/  et de Gestion - Vaud                                 
%//                                                                             
%// @title    M�moire de dipl�me: Colophon                                      
%// @context  Travail de dipl�me                                                
%// @author   Y. Chevallier <nowox@kalios.ch>                                   
%// @file     colophon.tex                                                      
%// @language ASCII/C                                                           
%// @svn      $Id$                
%///////////////////////////////////////////////////////////////////////////////
\Large\textbf{Colophon:}\par\normalsize
La qualit� du document est le fruit d'un long travail d'apprentissage du c�l�bre logiciel de composition \LaTeX prononc� \begin{IPA}"lAtEk\end{IPA}. La mise en page et le style du texte sont vivement inspir� de l'ouvrage: \emph{Gestion de projet avec Subversion} des �ditions \textsf{O'REILLY}. Le format \textsf{B5} (176x250mm) normalis� \textsf{ISO 216} a �t� choisi pour ses dimensions plus r�duites permettant toutefois de conserver des marges lat�rales plus grandes offrant au texte plus d'espace donc un aspect plus a�r�. 

Le lion en fin de document � �t� dessin� par \emph{Duane Bibby} 

L'entier du document � �t� re-lu par Ren� Asper  

La fonte du texte fait partie de la s�rie utopia con�ue par Robert Slimbach. Cette police de caract�re combine l'axe vertical et les contrastes entre les pleins et les d�li�s des fontes du XVIII si�cle avec des innovations contemporaines dans le d�tail de la forme et des traits. 


\ldots 
