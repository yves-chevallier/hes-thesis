%-------------------------------------------------+----------------------------%
%  _   _        _                            _    |       (Prof: B.Schneider)  %
% | | | |      |_|                          | |   |                            %
% | |_| |_____  _ _____           _    _  __| |   |                            %
% | |_  | ___ || |  _  \  _____  \ \  / // _  |   | Author:   Y.Chevallier     %
% | | | | ____|| | |_| | (_____)  \ \/ /( (_| |   |                            %
% |_| |_|_____)|_|___  |           \__/  \____|   |                            %
%                  __| | Haute Ecole d'Ingenieurs |                            %
%                 |___/  et de Gestion - Vaud     | FileType: ASCII/LaTeX2e    %
%------------------------------------------------------------------------------%
% Title: TRAVAIL DE DIPLOME                                                    %
%------------------------------------------------------------------------------%
% $Id: journal.tex 20 2005-11-15 23:00:29Z Canard $                            %
%------------------------------------------------------------------------------%
\chapter{LonTouch Protocol --- Protocole de communication}
Pour permettre la communication entre les diff�rents composants du syst�me, il a fallu �tablir un protocole intervenant aux couches 4 ou 5 du mod�le OSI. L'�tablissement d'un tel protocole n'est pas ais� les questions qu'il est n�cessaire de se poser sont multiples et pour chacune d'entres-elle il n'existe pas de solution absolue. L'entier de ce chapitre est consacr� aux d�marches entreprises pour aboutir � un protocole de communication ind�pendant du m�dia utilis�. 

\section{Cahier des charges initial}
\clearpage
\section{Description du protocol LonTouch}

\fig{datagramme}{Aper�u du datagramme LonTouch}

Type de donn�es � transmettre:\\
\begin{center}
\begin{tabular}{|c|l|p{10cm}|}
	\hline
	0 &IDENTIFY& Paquet d'identification permettant aux \TP de s'annoncer aupr�s du \SE.\\
	\hline	
	1 &PING& Il peut arriver qu'une machine cherche � conna�tre le temps d'une communication, les paquets de type PING permettent de calculer ce temps. \\
	\hline	
	2 &CONFIG& L'�change de donn�es concernant la configuration de l'installation v�hiculent par ces paquets\\
	\hline	
	3 &SVNT& Il s'agit du paquet qui sera le plus utilis�. Il permet de mettre � jour, de demander une valeur de variable r�seau\\
	\hline	
	4 &FTP& Le t�l�chargement des interfaces du \SE aux \TP passent par ce type de paquets \\
	\hline	
	5 &RSA& Un m�chanisme de chiffrage pourra �tre utilis�. Bien que LonTalk dispose d'une telle fonctionnalit�, cette derni�re n'est pas utilisable sur un r�seau Ethernet. \\
	\hline	
	6 &ACK& Les accus�s de r�ceptiosn v�hicules par l'interm�diaire de ces paquets\\
	\hline	
\end{tabular}
\end{center}

Les flags:
\begin{center}
\begin{tabular}{|c|l|p{10cm}|}
	\hline
	0 & ACK & Demande d'accus�\\
	\hline      	
	1 & ECD & Paquet chiffr�\\
	\hline      	
	2 & PR & Priorit�. Lorsque ce bit est activ� le paquet sera plac� en t�te de la file d'attente. Lorsque il est inhib�, il sera plac� en queue. \\
	\hline      	
	3 & NA & Non attribu�.\\
	\hline      	
	4-7 & PATTERN & Ce motif permet d'accroitre la reconnaissance d'un datagramme \LonTouch. 	\\
	\hline	
\end{tabular}
\end{center}                                    
\fig{ack}{Fonctionnement du mode accus�}
