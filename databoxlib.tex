%///////////////////////////////////////////////////////////////////////////////
%//  _   _        _                            _                                
%// | | | |      |_|                          | |                               
%// | |_| |_____  _ _____           _     _ __| |                               
%// | |_  | ___ || |  _  \  _____  \ \  / // _  |                               
%// | | | | ____|| | |_| | (_____)  \ \/ /( (_| |                               
%// |_| |_|_____)|_|___  |           \__/  \____|                               
%//                  __| | Haute Ecole d'Ingenieurs                             
%//                 |___/  et de Gestion - Vaud                                 
%//                                                                             
%// @title    M�moire de dipl�me: Avant propos                                  
%// @context  Travail de dipl�me                                                
%// @author   Y. Chevallier <nowox@kalios.ch>                                   
%// @file     avantpropos.tex                                                   
%// @language ASCII/C                                                           
%// @svn      $Id: avantpropos.tex 110 2005-12-19 23:55:45Z Canard $             
%///////////////////////////////////////////////////////////////////////////////
\chapter{Code source, librairie databox}

\lstset{
	language=C,
	basicstyle=\sffamily\tiny, 
	breaklines=true, 
	tabsize=3, 
	backgroundcolor=\color{shadecolor},
	numbers=left
}

En guise d'exemple de programme C, voici la librairie \textsf{databox.c}. Il s'agit d'une des librairie les plus petites permettant ainsi de faire partie des annexes de ce document.  
\lstinputlisting{src/databox.c}
